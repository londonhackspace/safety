\title{Fire Risk Assessment\\
    London Hackspace \\
    447 Hackney Road\\
    London E2 9DY
}


\date{10th May 2013}

\author{Russell Garrett}

\maketitle

\subsection{Introduction}

The London Hackspace is a community-run non-profit workspace run with
the object of fostering cross-disciplinary technical, scientific, and
artistic skills through social collaboration and education.

The space includes desk space and classrooms as well as facilities for
cooking, electronics, textiles, woodworking, metalworking, and amateur
science.

London Hackspace is funded almost entirely by membership subscriptions
from its members.

Overall responsibility for safety in the Hackspace is held by the
member-elected Board of Trustees of London Hackspace Ltd. The facilities
are run by volunteer members, with the Trustees only stepping in if
problems are encountered. London Hackspace has no employees.

\subsection{Building}

London Hackspace is located on the ground and basement floors of
447 Hackney Road, with a total floor area of around 6500 sq ft.
The fire alarm system is provided by the landlord.

The construction of the building is brick and concrete, and all internal
walls are plasterboard on wood studding.

\subsection{Sources of Ignition}

\subsubsection{Electrical}

All socket circuits at London Hackspace are protected by MCBs.
The majority of circuits are also protected by residual current devices.

Cooking applicances in the kitchen are all electric, so ignition risk is
low.

\subsubsection{Workshop}

The workshop contains several tools which constitute a risk of ignition:

\begin{itemize}
\item
  The laser cutter constitutes a high risk of ignition when used on
  certain materials. Users of the laser cutter are required to be
  trained, and are required to watch the machine during cutting. A \COtwo
  fire extinguisher is provided for extinguishing fires in the laser
  cutter, which is not considered part of the standard fire protection.
\item
  The welding and grinding tools produce a lot of sparks which represent
  a risk of ignition. Users of these tools are made aware of the risks
  and the location of firefighting equipment.
\end{itemize}
\subsubsection{Other}

Smokers are directed to the rear yard. Containers are provided for
cigarette butts.

\subsection{Sources of Fuel}

The fabric of the building is concrete. Sources of fuel on the ground
floor are primarily upholstery.

The workshop is used for woodworking and fuel exists there in the form
of wood and sawdust. As a volunteer-run workshop, housekeeping is a
continual issue. We are constantly working on improving this, and the
cleanliness of the workshop is currently checked weekly.

\subsection{People at Risk}

The London Hackspace is used by its members and their guests. There are
usually between 5 and 30 people in the space.

\subsection{Escape Routes}

The front door on the ground floor is a shopfront-style door and so
poses a security risk at night. During public events it is always open,
however it is often closed and shuttered while members are in the building.

In this case, two independent fire exits are still maintained, as there is
a second fire exit on the west side of the building.

Unmaintained emergency lighting is provided in the basement, but not on
the ground floor (which has windows overlooking the street).

\subsection{Protection Measures}

One 2kg \COtwo extinguisher is located by the emergency exits on the
ground floor. A 7kg dry powder extinguisher is located in the basement.
The seals on these are checked monthly.

A 2kg \COtwo extinguisher is located by the laser cutter, however this is
not considered as a protection measure as it is frequently used to
handle low-level fires.
